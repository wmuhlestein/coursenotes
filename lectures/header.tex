% Only the next five fields need to be edited.
\newcommand{\lecAuth}{Brent Corbin}
\newcommand{\scribe}{William Muhlestein}
\newcommand{\authEmail}{Lecturer Email}
\newcommand{\scribeEmail}{Scribe Email}
\newcommand{\course}{Physics 1B}

\address{School Information}

% Adds a hyperlink to an email address.
\newcommand{\mailto}[2]{\href{mailto:#1}{#2}}

% These commands set the document properties for the PDF output. Needs the hyperref package.
\hypersetup
{
    colorlinks,
    linkcolor={black},
    citecolor={black},
    filecolor={black},
    urlcolor={black},
    pdfauthor={\scribe <\mailto{\scribeEmail}{\scribeEmail}>},
    pdfsubject={\course},
	pdftitle={Physics 1B},
    pdfkeywords={},
    pdfstartpage={1},
}

% Includes a figure
% The first parameter is the label, which is also the name of the figure
%   with or without the extension (e.g., .eps, .fig, .png, .gif, etc.)
%   IF NO EXTENSION IS GIVEN, LaTeX will look for the most appropriate one.
%   This means that if a DVI (or PS) is being produced, it will look for
%   an eps. If a PDF is being produced, it will look for nearly anything
%   else (gif, jpg, png, et cetera). Because of this, when I generate figures
%   I typically generate an eps and a png to allow me the most flexibility
%   when rendering my document.
% The second parameter is the width of the figure normalized to column width
%   (e.g. 0.5 for half a column, 0.75 for 75% of the column)
% The third parameter is the caption.
\newcommand{\scalefig}[3]{
  \begin{figure}[ht!]
    % Requires \usepackage{graphicx}
    \centering
  \fbox{
      \includegraphics[width=#2\columnwidth]{#1}
  }
    %%% I think \captionwidth (see above) can go away as long as
    %%% \centering is above
    %\captionwidth{#2\columnwidth}%
    \caption{#3}
    \label{#1}
  \end{figure}}

% Includes a MATLAB script.
% The first parameter is the label, which also is the name of the script
%   without the .m.
% The second parameter is the optional caption.
\newcommand{\matlabscript}[2]
  {\begin{itemize}\item[]\lstinputlisting[caption=#2,label=#1]{#1.m}\end{itemize}}

% Example environment.
\newtheoremstyle{example}{\topsep}{\topsep} %
     {}%         Body font
     {}%         Indent amount (empty = no indent, \parindent = para indent)
     {\bfseries}% Thm head font
     {}%        Punctuation after thm head
     {\newline}%     Space after thm head (\newline = linebreak)
     {\thmname{#1}\thmnumber{ #2}\thmnote{ #3}}%         Thm head spec

   \theoremstyle{example}
   \newtheorem{example}{Example}[section]
